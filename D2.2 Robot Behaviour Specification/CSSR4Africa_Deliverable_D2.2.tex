% Template LaTeX document for CSSR4Africa Deliverables
% Adapted from documents prepared by EPFL for the RobotCub project
% and subsequently by the University of Skövde for the DREAM project
%
% DV 28/06/2023

\documentclass{CSSRforAfrica}

\usepackage[hidelinks,colorlinks=false]{hyperref}
\usepackage[titletoc,title]{appendix}
\usepackage{latexsym}
\usepackage{tabularx,colortbl}
\usepackage{verbatim} % for comments

\newcommand{\blank}{~\\}
\newcommand{\checkbox}{{~~~~~~~\leavevmode \put(-7,-1.5){  \huge $\Box$  }}}

\begin{document}
\input{epsf}

%%
%% SHOULD NOT NEED TO BE CHANGED BEFORE THIS POINT
%% ------------------------------------------------
%%

\deliverable{D2.2}               
\title{D2.2 Robot Behaviour Specification}  

\leadpartner{Carnegie Mellon University Africa}     
%%\partner{}                               

\revision{1.2}                         
\deliverabledate{31/12/2023}    
\submissiondate{31/12/2023}   
\revisiondate{30/01/2024}                  
\disseminationlevel{PU}
\responsible{D. Vernon }            


%%
%% Create the titlepage
%%

\maketitle
 

\section*{Executive Summary}
%===============================================================
\label{executive_summary}
%%\addcontentsline{toc}{section}{Executive Summary}
 
Deliverable D2.2 is concerned with the identification of the robot’s behaviour in the two use-case scenarios described in Deliverable D2.1 Use Case Scenario Definition, Sections 3 and 4 of which define the robot actions, component movements and sensory cues, and the associated sensory-motor process specified in the use case interactions. 

The purpose of this deliverable is simply to identify the ROS nodes that will  implement these actions. It also identifies the ROS packages in which the ROS nodes will be implemented.  We defer the detailed specification of these nodes to Deliverable D3.1 System Architecture Design, i.e., the exact functional description, the topics to which the nodes will subscribe, the topics on which the nodes will publish messages, the services that will be advertized and served, and the services that will be invoked.

Together with Deliverable D3.1, Deliverable D2.2 provides the requirements for work package WP5 on robot behaviours, complementing and augmenting the detailed specification already provided in the work plan.   

The dynamics of the interaction between the visitor and the robot will be specified in Deliverable D5.4.2 Scenario Script Language, and implemented in the {\small \verb+scriptInterpreter+} node in Deliverable D5.4.3 Scenario Script Interpreter.



\newpage
 
 
%\graphicspath{{./figs/}}
\pagebreak
\tableofcontents
\newpage


\section{Implementation of Robot Actions}
%===============================================================
Deliverable D2.1 Use Case Scenario Definition, version 1, identifies the set of baseline robot actions that are invoked when the robot interacts with the visitor, either in its role as a receptionist or as a lab tour guide.   These are summarized in Table \ref{table:nodes} below, along with the corresponding of ROS nodes that implement the robot actions, and the ROS packages in which the nodes are implemented.  Each node has a set of  configuration parameters that, suitably chosen, allows it to produce one or more required robot actions. 

Deliverable D3.1 System Architecture Design provides a detailed specification of each ROS node. It specifies the node configuration parameters that are read from the associated configuration file, the data that are read from the associated input file, the data that are written to an associated output file, the ROS topics to which the node subscribes for input,  the ROS topics on which the node publishes output, the services that will be advertized and served, and the services that will be invoked.

The dynamics of the interaction between the visitor and the robot, implemented with these  ROS nodes and those identified in Deliverable D2.3, will be specified in Deliverable D5.4.2 Scenario Script Language, and implemented in the {\small \verb+scriptInterpreter+} node in Deliverable D5.4.3 Scenario Script Interpreter.

\begin{table}[thb]
\begin{center}
\begin{tabular}{|l l l|}
\hline 
Robot Actions                                & ROS Package                                                  &  ROS Node       \\
\hline
{\small Animate behaviour }	                                  & {\small \verb+cssr_system+}     & {\small \verb+animateBehaviour+} \\
{\small Deictic gesture }                                          & {\small \verb+cssr_system+}       & {\small \verb+gestureExecution+} \\
{\small Display dialogue }                                       & {\small \verb+cssr_system+}       &  {\small \verb+scriptInterpreter+}  \\
{\small Iconic gesture}                                             & {\small \verb+cssr_system+}      & {\small \verb+gestureExecution+}  \\
{\small Locomotion }                                                 & {\small \verb+cssr_system+}     & {\small \verb+robotNavigation+}  \\
{\small Look up knowledge-base }                           & {\small \verb+cssr_system+}      &  {\small \verb+knowledgBase+}  \\
{\small Navigation  }                                               & {\small \verb+cssr_system+}       & {\small \verb+robotNavigation+}  \\
{\small Rotate head to centre gaze on the visitor }   &  {\small \verb+cssr_system+}     & {\small \verb+overtAttention+} \\
{\small Rotate head toward the office}                      &  {\small \verb+cssr_system+}    & {\small \verb+overtAttention+} \\
{\small Rotate torso to face visitor and adjust gaze}  &  {\small \verb+cssr_system+}  & {\small \verb+overtAttention+} \\
{\small Speech synthesis}                                         &  {\small \verb+cssr_system+}    & {\small \verb+textToSpeech+} \\
\hline
\end{tabular}
\end{center}
\caption{The ROS packages and nodes that implement the robot actions in the CSSR4Africa  system.}
\label{table:nodes}
\end{table}
 


%\newpage
%\bibliographystyle{unsrt}
%================================================================
%\bibliography{cognitive_systems.bib}                                     % REPLACE with correct filename
%\addcontentsline{toc}{section}{References}



\pagebreak
\section*{Principal Contributors}
%===============================================================
\label{contributors}
\addcontentsline{toc}{section}{Principal Contributors}
The main authors of this deliverable are as follows (in alphabetical order).
\blank
~
\blank
David Vernon, Carnegie Mellon University Africa.\\   
  

\newpage
\section*{Document History}
%================================================================
\addcontentsline{toc}{section}{Document History}
\label{document_history}

\begin{description}

\item [Version 1.0]~\\
First draft. \\
David Vernon. \\                          
31 December 2023.

\item [Version 1.1]~\\
Adopted British English spelling of ``behaviour'' rather than American English because South African English is closer to British English. \\
David Vernon. \\                          
2 January 2024.

\item [Version 1.2]~\\
Rationalized the ROS nodes for text-to-speech to align with the work plan and Deliverable 3.1 System Architecture. Instead of four TTS nodes --- English, Kinyarwanda,  isiZulu, and an integration version --- these is now just one {\sf textToSpeech} node which calls three functions. \\
David Vernon.\\
30 January 2024.

\end{description}

\end{document}

